\documentclass{fortythree_resume}

% Preamble
\usepackage{fancyhdr}
\pagestyle{fancy}
\fancyhf{}

\usepackage{fontawesome}

\begin{document}
% Name Section
\namesection{Fortythree}{Shiaohongtu}
\sectionsep

\begin{minipage}[t]{0.33\textwidth}

% Education Section
\section{Education}
\subsection{The University of Texas at Dallas}
\descript{B.S. in Computer Science}
\emph{December 2019 | GPA 3.7/4.0}
\sectionsep

% Contact Info Section
\section{Contact Info}
\faGithub \hspace{0em} github.com/BraneXZ\\
\faLinkedinSquare \hspace{0em} linkedin.com/in/43shiaohongtu\\
\faMobile \hspace{.2em} (832) 851-9568\\
\faEnvelope \hspace{0em} washingtonshiao@gmail.com
\sectionsep

% Skills Section
\section{Skills}
\subsection{Programming Languages}
Python \textbullet{} Node.Js \textbullet{} Java \textbullet{} C\# \\
JavaScript \textbullet{} HTML \textbullet{} \LaTeX \\
Scala
\subsection{Databases}
SQL \textbullet{} Oracle \textbullet{} SQLite\\ 
MongoDB \textbullet{} AzureCosmosDB \\
AmazonRDS \textbullet{}AmazonAthena \\
Databricks \textbullet{} BigQuery
\subsection{Languages}
English, \emph{Fluent}\\
Mandarin Chinese, \emph{Fluent}\\
\subsection{Cloud-Based Technologies}
AWS \textbullet{} Docker \textbullet{} Spark
\subsection{Automation}
Apache Airflow
\sectionsep

% Coursework Section
\section{Coursework}
Machine Learning\\
Probabilistic Graphical Models\\
Network Security\\
Database Design\\
Data Structures \& Algorithms\\
Operating Systems Concepts\\
Convolutional Neural Networks
\sectionsep

% Certification Section
\section{Certification}
Tensorflow Developer Certificate

\end{minipage}
\hfill
\begin{minipage}[t]{0.66\textwidth} 

% Experience Section
\section{Experience}
\company{Sigmoid Analytics}{Associate Software Engineer}
\location{Remote}{September 2020 --- Now}
\sectionsep
\begin{tightemize}
	\item \itemdescript{Create automation scripts for cloud-based \textbf{ETL} pipelines using \textbf{AWS} services and \textbf{airflow}. One project involves survey data landing in an S3 bucket which will trigger a \textbf{Lambda} function that starts an airflow DAG. Another would be creating a \textbf{docker image} containing scripts that performs machine learning training and prediction process using \textbf{AWS SageMaker} and \textbf{Kubernetes} pods}
	\item \itemdescript{Lead and guide team members in their development within the project. Improves overall code quality and efficiency between different teams}
\end{tightemize}
\sectionsep

\company{File \& Serve Express}{Junior Software Developer}
\location{Irving, Texas}{June 2019 --- March 2020}
\begin{tightemize}
	\item \itemdescript{Exposed \textbf{REST APIs} on a legacy system and utilized its functionality on a GUI using \textbf{C\#}. This improved efficiency and performance in manipulating data by the AppOps team}
	\item \itemdescript{Developed web-based \textbf{C\# .NET Core} applications using Visual Studio}
	\item \itemdescript{Created unit tests for all code submitted and extend test coverage for existing legacy code}
	\item \itemdescript{Collaborated with the QA analyst team to understand application functionality and automate test cases}
	\item \itemdescript{Maintenance and support of electronic filing service provide products, including front-end UI improvements/updated with back-end integration and performance upgrades}
	\item \itemdescript{Mentored and assisted others in the team who are less familiar with the code base to establish knowledge on the product, which increased development speed and maximized sprint cycle efficiency}
\end{tightemize}
\sectionsep

% Projects Section
\section{Projects}
\project{Tensorflow Practice}{July 2020 --- August 2020}
\begin{tightemize}
	\item \itemdescript{Built multiple different neural network models using \textbf{Tensorflow} for different tasks}
	\item \itemdescript{Tasks includes image classification, text sentiment analysis, time series forecasting, and text generation}
\end{tightemize}
\sectionsep

\project{Deep Learning on Minesweeper}{May 2020 --- September 2020}
\begin{tightemize}
	\item \itemdescript{Implemented game logic and machine learning models using \textbf{python} and front end UI using \textbf{Flask}}
	\item \itemdescript{Agents learned completely through self-play, with no past human experiences}
	\item \itemdescript{Utilized the \textbf{Keras} and \textbf{Tensorflow} library}
	\item \itemdescript{Models created using \textbf{convolutional neural network} by applying different techniques such as \textbf{batch normalization}, \textbf{dropout}, \textbf{padding}, and \textbf{residual connections}}
	\item \itemdescript{Created multiple reinforcement learning agents using \textbf{policy gradient, actor-critic} and \textbf{q-learning} method}
\end{tightemize}
\sectionsep

\project{Intelligent Tutoring System}{January 2019 --- May 2019}
\begin{tightemize}
	\item \itemdescript{Web application that captures student's attentiveness and emotion during lectures}
	\item \itemdescript{Back-end machine learning model implemented using \textbf{python}, \textbf{mongoDB} as database and \textbf{node.js} for front-end UI}
	\item \itemdescript{Worked in a team of 5 people including an adviser where tasks were split among the team in weekly meetings}
\end{tightemize}
\sectionsep

\end{minipage}

\sectionsep
\end{document}
